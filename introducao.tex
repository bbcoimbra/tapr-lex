\section{Introdu��o}
O processo inicial de compilar um programa consiste em ler um arquivo
fonte, que � basicamente uma sequ�ncia de caracteres, e agrup�-los
em unidades significativas, chamadas \Marks. Este processo �
denominado de \AL.

Desta forma, podemos considerar o processo de \AL{} equivalente ao
processo de soletrar uma palavra. Verificamos caractere por caractere
at� podermos classific�-lo em uma categoria pr�-definida, gerando uma
\Mark {} (tamb�m conhecida como token).

As \Marks {} podem ser classificadas como far�amos em linguagem natural.
Em Portugu�s, por exemplo, para as \Marks, temos as classes Verbo, Substantivo,
Artigo etc.

O processo de \AL {} efetua outras opera��es al�m do reconhecimento das
\Marks. Ele pode, por exemplo, pr�-calcular o valor dos literais
num�ricos, reconhecendo a string ``3.14159'' como o n�mero 3.14159.
