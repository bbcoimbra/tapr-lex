\section{Reconhecimento de Padr�es}
Um analisador l�xico utiliza basicamente dois m�todos  para
reconhecer os padr�es de caracteres determinados por uma linguagem:
Express�es Regulares e Aut�matos Finitos.

\subsection{Express�es Regulares}
Express�es Regulares representam padr�es de cadeias de caracteres. Uma
express�o regular \emph{r} � definida pelo conjunto de cadeias de
caracteres que ela casa.

A esse conjunto denominamos \emph linguagem gerada pela express�o regular
\cite{louden97-pt}.

Precisamos diferenciar os caracteres que formam o padr�o dos
caracteres reconhecidos pelo padr�o. Para isso usaremos a seguinte
nota��o:

\begin{itemize}
	\item \emph{/abc/} para representar o padr�o que dever� ser reconhecido;
	\item \emph{``abc''}  para representar a cadeia de caracteres em que
		o padr�o ser� aplicado.
	\end{itemize}

Exemplos
\begin{itemize}
	\item a express�o regular /teste/ casa com a palavra \emph{teste} no texto ``Vamos fazer um
		teste com Express�es Regulares?''
\end{itemize}

\subsubsection{Opera��es com Express�es Regulares}
\begin{description}
	\item[Escolha] se \emph{r} e \emph{s} s�o express�es regulares,
		/r\textbar s/ representa a uni�o da express�o regular \emph{r} e da
		express�o regular \emph{s}. Como exemplo considere a express�o
		regular /a\textbar b/. Essa express�o casa com os caracteres ``a''
		ou ``b''.
	\item[Concatena��o] se \emph{r} e \emph{s} s�o express�es regulares,
		/rs/ casa com qualquer express�o que case com \emph{r} seguida de
		uma express�o que case com \emph{s}. Como exemplo, considere a
		express�o regular /ab/. Essa express�o casa com a sequ�ncia de
		caracteres ``ab''.
	\item[Repeti��o] 
\end{description}
